\documentclass[%
	11pt,
	a4paper,
	utf8,
	%twocolumn
		]{article}	

\usepackage{style_packages/podvoyskiy_article_extended}


\begin{document}
\title{Заметки по машинному обучению и анализу данных. Том 2}

\author{\itshape Подвойский А.О.}

\date{}
\maketitle

\thispagestyle{fancy}

Здесь приводятся заметки по некоторым вопросам, касающимся машинного обучения, анализа данных, программирования на языках \texttt{Python}, \texttt{R} и прочим сопряженным вопросам так или иначе, затрагивающим работу с данными.


\shorttableofcontents{Краткое содержание}{1}

\tableofcontents

\section{Приемы работы с библиотеками Gym и Ecole}

\subsection{Gym}

Функция окружения (environment) \texttt{step} возвращает четыре значения:
\begin{itemize}
	\item \verb|observation| (object):  это объект, специфичный для окружающей среды и представляющий результат наблюдения за этой средой (например, состояние доски в настольной игре),
	
	\item \verb|reward| (float): вознаграждение, полученное за предыдущее действие. Масштаб варьируется в зависимости от среды, но цель всегда в том, чтобы сделать суммарное вознаграждение как можно больше,
	
	\item \verb|done| (boolean): флаг завершения эпизода. Многие (но не все) задачи разделены на четко определенные эпизоды, и \texttt{done = True} указывает на то, что эпизод завершился (например, мы потеряли последнюю жизнь в игре),
	
	\item \verb|info| (dict): диагонстическая информация, полезная для отладки.
\end{itemize}

Это просто реализация классического цикла <<агент -- среда>>. На каждом шаге агент совершает то или иное действие и среда возвращает наблюдения (observation) и вознаграждение (reward).

Процесс запускается вызовом функции \verb|reset()|, которая возвращает первое приближение \texttt{observation}.  
\begin{lstlisting}[
style=ironpython,
numbers=none
]
import gym
env = gym.make('CartPole-v0')
for i_episode in range(20):
    observation = env.reset()
    for t in range(100):
        env.render()
        print(observation)
        action = env.action_space.sample()
        observation, reward, done, info = env.step(action)
        if done:
            print("Episode finished after {} timesteps".format(t+1))
            break
env.close()
\end{lstlisting}

В этом примере мы отбирали случайные действия из пространства действий среды. Каждая среда поставляется с атрибутами \verb|action_space| и \verb|observation_space|. Эти атрибуты имеют тип \verb|Space| и описывают формат допустимых действий и наблюдений
\begin{lstlisting}[
style=ironpython,
numbers=none
]
import gym

env = gym.make("CartPole-v0")
print(env.action_space)  # Discrete(2)

print(env.observation_space)  # Box([-4.8000002e+00 -3.4028235e+38 -4.1887903e-01 -3.4028235e+38], [4.8000002e+00 3.4028235e+38 4.1887903e-01 3.4028235e+38], (4,), float32)
\end{lstlisting}

Пространство \texttt{Descrete} описывает фиксированный диапазон неотрицательных чисел, так что в данном случае допустимыми действиями будет 0 или 1. Пространство \texttt{Box} представляет $ n $-мерный ящик, так что в данном случае допустимыми наблюдениями будут 4-мерные массивы.

\subsection{Ecole}

Полезный ресурс о специальных приемах работы с задачами линейного программирования в частично-целочисленного постановке \url{https://www.gams.com/37/docs/UG_LanguageFeatures.html?search=sos1}

Полезный ресурс по математической оптимизации \url{https://scipbook.readthedocs.io/en/latest/}

\subsubsection{Observations}

Класс \texttt{ecole.observation.NodeBipartiteObs}: двудольный граф наблюдений для узлов branch-and-bound дерева. Оптимизационная задача представляется в виде гетерогенного двудольного графа. Между переменной и ограничением будет существовать ребро, если переменная присутствует в ограничении с ненулевым коэффициентом.

Метод \texttt{reset()} в Ecole принимает в качестве аргумента экземпляр проблемы. 



\listoffigures\addcontentsline{toc}{section}{Список иллюстраций}

% Источники в "Газовой промышленности" нумеруются по мере упоминания 
\begin{thebibliography}{99}\addcontentsline{toc}{section}{Список литературы}
	\bibitem{lutz:learningpython-2011}{\emph{Лутц М.} Изучаем Python, 4-е издание. -- Пер. с англ. -- СПб.: Символ-Плюс, 2011. -- 1280~с. }
		
	\bibitem{beazley:python-2010}{\emph{Бизли Д.} Python. Подробный справочник. -- Пер. с англ. -- СПб.: Символ-Плюс, 2010. -- 864~с. }
\end{thebibliography}

\end{document}
